\documentclass[ngerman]{scrartcl}
\usepackage{amsmath,amsthm,amssymb}
\usepackage[T1]{fontenc}
\usepackage[utf8]{inputenc}
\usepackage{lmodern}

\title{Transaktionale Informationssysteme \\ SoSe19}

\begin{document}

\maketitle
\tableofcontents
\newpage
\begin{abstract}
\end{abstract}

\section{1. Vorlesung}
Foliensatz 1
\subsection*{Orga}
\begin{itemize}
  \item \textbf{Vorlesung} Di, 14:15-15:45, H11
  \item \textbf{Übung} Mo, 13-14
  \item \textbf{Prüfung} mündlich 16.06. und 22.10.
\end{itemize}
\subsection*{Motivation}
Bei vielen, kurzen Transaktionen (Änderungen) darf die Datenbasis nicht zerstört werden
\begin{itemize}
  \item Rollback
  \item Administration der Aktionen auf der Datenbasis
  \item $\Rightarrow$ Datenkonsistenz
\end{itemize}
\paragraph*{Konsistenz}
Bewahrung der Korrektheit Daten im Fehlerfall

\paragraph*{Generizität}
Abstraktion von Szenarien

\subsection*{Paralleler Zugriff Beispiel 1.1 (Folie 12)}
Naive Parallelverarbeitung sorgt zum Konflikt

\paragraph*{Optimistischer Ansatz}
Laufen lassen, bis ein Fehler Auftritt

\paragraph*{Pessimistische Ansatz}
Zugriff blockieren

\subsection*{Fehlerhafte Ausführung Beispiel 1.2 (Folie 13)}
Prozess wird durch Fehler unterbrochen

\paragraph*{Rollback}
Sollten nicht alle Aktionen ausführbar sein, nicht ausführen (Komplett oder gar nicht)

\subsection*{Verteiltes Datensystem Beispiel 1.3 (Folie 14)}
Verschiedene Datenbestände nicht korrekt synchronisiert (zB Client- und Serverwarenkorb), Datensysteme sind verschieden und unahängig voneinander (heterogen und autonom)

\paragraph*{Transaktionale Eigenschaften}
\begin{itemize}
  \item Synchronisierung von Client und Serverinformationen
  \item Verifikation des Abschlusses einer Transaktion
\end{itemize}

\subsection*{Beispiel 1.4 (Folie 19)}
Gesamte Aktion muss erfolgreich sein: Schlägt eine Transaktion im Block fehl, wirf eine Fehlermeldung (zB Prüfungsanmeldung und Bestätigung)

\subsection*{Workflow Management}
Spezifikation von Workflows 
\begin{itemize}
  \item Wer bekommt welche Rolle
\end{itemize}

\paragraph*{Workflow}
\begin{itemize}
  \item Geschäftsprozess (zB Beschaffung, Reiseplanung) 
  \item Langlebig
\end{itemize}
\paragraph*{Aktivität}
Teile eines Workflows, die von verschiedenen Akteuren augeführt werden

\subsection*{Architekturen}
\paragraph*{Einfache Server Struktur (Folie 27)}
Data Server: Datendatendarstellung
\begin{itemize}
  \item Gekapselt in Objekten (Request, Reply)
  \item Ungekapselt als Tupel 
\end{itemize}
\paragraph*{Föderierte Systeme}
\begin{itemize}
  \item Alte Systeme müssen mit neuen Systemen kooperieren
\end{itemize}

\subsection*{Transaktionsmanagement}
\paragraph*{ACID (Folie 30)}
\begin{itemize}
  \item Atomarität: Ganz oder gar nicht
  \item Consistenz: Konsistenzerhaltung, waren die Daten Konsistent vor der Transaktion, sind sie es auch danach
  \item Isolation: Transaktionen beeinflussen sich nicht gegenseitig
  \item Dauerhaftigkeit: Wenn Transaktion erfolgreich, so ist sie in der Datenbank vorhanden 
\end{itemize}
\paragraph*{Anforderungen and Transaktionsmanagement (Folie 31)}
\begin{itemize}
  \item Concurrency Control
  \item !nachgucken!
\end{itemize}

\paragraph*{Aufbau (Folie 32)}
\begin{itemize}
  \item Transaktionsmanagement sorgt für Synch der Zugriffe
  \item Datenbank-Cache: Lesen und Bearbeiten der Daten im DB-Cache. Schreiben geschieht später
  \item DB Seiten (Folie 37)
\end{itemize}

\end{document}
