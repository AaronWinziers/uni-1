\documentclass[ngerman]{scrartcl}
\usepackage{amsmath,amsthm,amssymb}
\usepackage[ngerman]{babel}
\usepackage[utf8]{inputenc}
\usepackage{graphicx}


\title{Formale Sprachen B\\ Beschreibungskomplexit{\"a}t}
\author{Benedikt L{\"u}ken-Winkels}
\begin{document}

\maketitle
\tableofcontents
\newpage
\section{Allgemeines}
\subsection{Chomsky Hierarchie}
\begin{itemize}
\item Typ-0: rekursiv (aufzählbar) (beliebige formale Grammatik)
\item Typ-1: Kontextsensitive Grammatik
\item Typ-2: Kontextfreie Grammatik (CFG)
\item Typ-3: Reguläre Grammatik
\end{itemize}
\subsection{Begriffsklärung}
\subsubsection{rekursiv (aufzählbar) Typ-0}
\begin{itemize}
\item $L$ ist \textit{rekursiv aufzählbar} bzw. \textit{semientscheidbar}, wenn es eine Turingmaschine gibt, die alle $w \in L$ akzeptiert, aber keine Wörter, die nicht in $L$ liegen.
\item $L \subseteq \Sigma^{*}$ ist \textit{rekursiv} bzw. \textit{entscheidbar}, wenn es eine Turingmaschine gibt, die für jede Eingabe $w \in \Sigma^{*}$ hält und jedes $w$ genau dann akzeptiert, wenn $w \in L$.
\end{itemize}
Beispiel: Das Halteproblem ist rekursiv aufzählbar, aber nicht rekursiv. \\
\begin{align*}
Beweisskizze
\end{align*}

\subsubsection{kontextsensitiv Typ-1}
\begin{itemize}
\item  $G$ ist \textit{kontextsensitiv}, wenn die Regeln der Form $\alpha A \beta \rightarrow \alpha \gamma \beta $ sind (wobei $\gamma$ entweder NT oder T sein muss; $\alpha, \beta$ dürfen leer sein) oder $S \rightarrow \varepsilon $ (dann darf $S$ allerdings nicht auf eine rechten Regelseite auftauchen), also
\begin{itemize}
\item NT-Symbole auf der linken Regelseite,
\item keine Produktionsregel außer das Startsymbol darf  $\varepsilon$ erzeugen.

\end{itemize}
\end{itemize}

\subsubsection{kontextfrei Typ-2}
\begin{itemize}
\item $G$ ist \textit{kontextfrei}, wenn die Regeln der Form $A \rightarrow \alpha $ (wobei $A \in NT$ und $\alpha$ eine beliebige Folge von NT und T) oder $S \rightarrow \varepsilon$  (dann darf $S$ allerdings nicht auf eine rechten Regelseite auftauchen) sind 
\end{itemize}


\subsection{Pumping Lemma}

\subsection{Push Down Automate (PDA)}

\section{1. Vorlesung}
\subsection{Ziel von Beschreibungskomplexität}
\begin{itemize}
\item Wie kompakt können 'Gegenstände' oder 'Objekte' ausgedrückt werden?

\begin{itemize}
\item DEA als Beispiel mit kleinstmöglicher Zustandsmenge
\item Minimierungsalgortihmen für Automaten
\item Datenkomprimierung (Lempel-Ziv-Welch-Algorithmus, verlustfreies Komprimierungsverfahren)
\end{itemize}
\item[$\Rightarrow$] knappe Beschreibung für Objekte für effizientere Arbeit
\end{itemize}
\subsection{Beschreibungssystem}
Ein \underline{Beschreibungssystem}  $S$ besteht aus einer Menge endlicher \underline{Deskriptoren}, sodass jeder Deskriptor $D\in S$ eine formale Sprache $L(D)$ beschreibt. Aus $D$ kann man $alph(D)$ ablesen, sodass $L(D) \subseteq (alph(D))*$. Die durch $S$ beschriebene Sprachfamilie $\ell(S)$ umgekehrt $L:S(L)$ beschreibt $L$.
\\
Es existieren verschiedene Beschreiber für eine Sprache, eventuell abzählbar unendlich viele Möglichkeiten.
\begin{itemize}
\item Umsetzung einer Aufgabe/Funktion hat beliebig viele Implementierungen
\item Verschiedene reguläre Ausdrücke für die selbe Sprache
\end{itemize}

\subsection{Beschreibungsmaße}
Natürliches Maß:
\# Bits einer Beschreibung \\
 $\rightarrow$ Frage: Wie wird kodiert? \\
Alternativ: Betrachte die Struktur der Deskriptoren \\
Ist $S_{mass}$ ein Beschreibungsmaß für $S_{sys}$, so meint \\$S_{mass}(L)=min\{ S_{mass}(D) | D \in S_{sys}, L(D) = L$

\subsubsection{\textbf{Beispiel} reguläre Ausdrücke}
\begin{itemize}
\item $\emptyset, \lambda, a\in \sum$ reguläre Ausdrücke
\item Wenn $r,s$ reguläre Ausdrücke, so auch $(r+s),(r \cdot s),(r*)$
\end{itemize}

\textbf{Größenmaße}
\begin{itemize}
\item Länge des Strings $|(\emptyset)*| = 4 > 1 = |\lambda|$
\item $rpn(n)$ reversed polish notation
\begin{itemize}
\item $r = ((0+((1 \cdot 0)*)) \cdot (1 + \lambda))$
\item[$\rightarrow$] $rpn(r) = |010\cdot * + 1 \lambda + \cdot| = 10$
\end{itemize}
\item $a-width(r)$: Anzahl von Vorkommen von Zeichen aus $\sum$ in $r$
\item Ressourcen: Nonterminalsymbole zählen zum Beispiel
\end{itemize}

\section{2. und 3. Vorlesung}
\subsection{Typ-0 Grammatiken}
$G = (N,T,P,S)$\\
\textit{N} = non-Terminalsymbole; \textit{T} = Terminalsymbole; \textit{S} Startsymbole \\
$P \subseteq (N \cup T)* N (N \cup T)* \times (N \cup T)*$\\
Wenn alle Regeln bis auf Eine kontextfrei sind können alle rekursiv aufzählbaren formalen Sprachen dargestellt werden.
Typ-0 Grammatiken können durch durch Turingmaschinen dargestellt werden.\\
Natürliche Maße für Turingmaschinen
\begin{itemize}
\item Laufzeit als Maß kann unentscheidbar sein, weil dynamisch
\item Bandalphabet
\item Anzahl der Bänder
\item Grad des Nichtdeterminismus, zB wie oft gibt es nichtdeterministische Übergänge?
\end{itemize}
Fragen: In wie weit können Maßzahlen eingeschränkt werden, ohne das Modell zu ändern?\\
Was sagen die Maße über die Mächtigkeit der Maschine aus?
\subsubsection{Universelle Turingmaschine mit 2 Zuständen}
Die Übergänge teilen sich in eine Kopierphase und einen Simulationszyklus: '+' bedeutet, dass der Simulationszyklus läuft; '$\alpha$' beendet die Kopierphase.\\
Zu den Regeln: Es existiert eine Bouncing-Regel zwischen den Regeln 1, 3 und Regeln 2,4, die für die Codierung auf dem Band sorgt.
\subsection{Begriffseinführungen}
Ziel ist eine minimale Beschreibung von Typ-0 Grammatiken
\subsubsection{1-a-Transitor}
Ein 1-a-Transitor $\tau = (Q, \sigma, \delta, H, q_{I}, q_{F})$ ist ein endlicher übersetzender Automat.\\
$\sigma$ = Eingabe, $\delta$ = Ausgabe, Q = Zustandsmenge, $H \subseteq Q \times \sigma \times \delta \times Q \Rightarrow$ nicht deterministisch.
\begin{itemize}
\item Allgemeiner, als Mealy und Moore Automaten
\end{itemize}

\subsubsection{G-Systeme}
Ein $G$-System $G = (N,T,P,S)$ wobei $P = (K,V,V,H,q_{I}, q_{F})$ ein 1-a-Transitor ist und das Eingabe gleich dem Ausgabealphabet ist mit $V=N\cup T$.\\
Typ-0-Sprachen können durch G-Systeme simuliert werden.
\end{document}













%========
%
%
%
%
%========