\documentclass[ngerman]{scrartcl}
\usepackage{amsmath,amsthm,amssymb}
\usepackage[T1]{fontenc}
\usepackage[utf8]{inputenc}
\usepackage{lmodern}

\title{Approximationsalgorithmen \\SoSe 2019}

\begin{document}

\maketitle
\tableofcontents
\newpage
\begin{abstract}

\end{abstract}

\section{1.Vorlesung}
Foliensatz 1
\subsection{Orga}
\begin{itemize}
  \item \textbf{Sprechstunde} Do, 13-14 Uhr
  \item \textbf{Vorlesung} Di, 12:15-13:45
  \item \textbf{Übung} Di, 8:15-9:45 Uhr (erster Termin 16.04.)
  \item \textbf{Prüfung} Mündl Prüfung
\end{itemize}

\subsection{Einführung}
\subsubsection*{Motivation}
\begin{itemize}
  \item wenn P $ \neq $ NP, kan man keinen guten oder schnellen Algorithmus schreiben
  \item Zeigt man, dass ein Problem NP-schwer ist, kann kein schneller Algorithmus geschrieben werden  
\end{itemize}
$\Rightarrow$ Heuristische Verfahren (keine mathematische Garantie). Warum funktionieren die Heuristiken so gut?
Herangehensweisen
\begin{itemize}
  \item Greedy Verfahren
  \item Randomisierte Verfahren: finden der Lösung mit hoher Wahrscheinlichkeit
  \item Parametrisierte Verfahren: exakte Lösungen und Versuch, den exponentiellen Teil gering zu halten
  \item Näherungsverfahren: Heuristiken mit Leistungsgarantie
\end{itemize}
Klasse von Problemen die zur Betrachtung stehen. \\
\textbf{Quatrupel} $ (I\rho, S\rho, m\rho , opt\rho) $ zur Beschreibung eines Optimierungsproblems
\begin{itemize}
  \item $ I\rho $: geeignete Instanz eines Problems, genauer: "geeignet binär-codierte formale Sprachen".
  \item $ S\rho $: Bildet auf Menge der möglichen Lösungen ab
  \item $ m\rho $: x Instanz und y eine Lösung. Abbildung auf Maßzahl
  \item $ opt\rho $: Möglichst kleines Ergebnis oder möglichst großes
\end{itemize}
\begin{itemize}
  \item $ S*\rho: I\rho \rightarrow $ Menge der bestmöglichen Lösungen
  \item $ m*\rho $ Wert oder Grenzwert einer bestmöglichen Lösung
  \item * bedeutet idR bestmöglich
\end{itemize}
$\Rightarrow$ \textbf{Ziel}: Leistungsgröße (Folie 15) ist 1, wenn Lösung optimal ist

\subsubsection*{Beispiel: Knotenüberdeckung}
Möglichst wenige Knoten, um alle Kanten abzudecken
\begin{itemize}
  \item Zuordnung zu den Optimierungsparametern Folie 17
  \item Verschiedene Beobachtungen zur Optimierung
  \begin{itemize}
    \item Zwei Knoten im Dreieck gehören dazu
    \item Bei Knoten mit Grad 1 wird immer der Nachbar genommen
    \item 
  \end{itemize}
  \item Auswählen eines Knotens bedeutet, dass diese Teile abgeschnitten werden
  \item $\Rightarrow$ Vereinfachung des Graphen, zB neue Grad 1 Knoten
\end{itemize}

\paragraph*{Greedyverfahren, GreedyVC (Folie 23)}
\begin{itemize}
  \item Änderung der Grade bei Durchführung
  \item Problem: Implementierung der Kantenlöschung (Kopieren des Graphen bei jeder Iteration nötig?)
  \item Folie 24: Lösung insofern (inklusions-) minimal, als dass das Entfernen eines Knotens keine andere Lösung zulässt
\end{itemize}

\paragraph*{Suchbaumverfahren, Entscheidungsproblem (Folie 25)}
Liefert exakte Lösungen
\begin{itemize}
  \item Zusätzlicher Parameter $ k $ ("Budget")
  \item Zwei Abbruchskriterien:
  \begin{itemize}
    \item Alle Kanten abgedeckt
    \item Nicht alle Kanten abgedeckt, aber $ k=0 $
  \end{itemize}
  \item Suchbaum im worst-case ein vollständiger Binärbaum, \textbf{aber} höchsten $ 2^{k} $ Schritte im Baum, da die Tiefe durch $ k $ begrenzt ist
\end{itemize}
\paragraph*{Näherungsverfahren (Folie 30)}
Suchbaumverfahren ohne Fallunterscheidung. (Faktor 2-Approximations-Verfahren)
\begin{itemize}
  \item Bei jeder Kante muss einer der Knoten in die Überdeckung 
  \item Lokaler Fehler höchsten Faktor 2
  \item Zufall bei der Auswahl der Kanten kann zum Vorteil sein
\end{itemize} 

Näherung gibt Schranke für die minimale Lösung dadurch, dass Heuristik eine Faktor 2 Lösung zeigt.
$\Rightarrow$ (Folie 31) Lösung mit 22 Knoten zeigt eine optimale Lösung mit 11 Knoten

\subsubsection*{Beispiel: MAXSAT (Folie 32)}
$ m\rho $ = Anzahl der Klauseln, die die Formel erfüllen
\paragraph*{Einfacher Ansatz}
\begin{itemize}
  \item Alles 0 und alles 1 setzen, dann das bessere Ergebnis zurückliefern
  \item $\Rightarrow$ liefert 2-Approximation
\end{itemize}

\subsubsection*{Beispiel: Unabhängige Knotenmengen (Folie 34)}
Sehr schwer approximierbar

\subsubsection*{Beispiel: Unabhängige Kantenmengen (Folie 35)}
Lösung in Polinomialzeit, um eine untere Schranke für die Knotenüberdeckung zu finden


\end{document}
