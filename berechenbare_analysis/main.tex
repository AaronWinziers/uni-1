\documentclass[ngerman]{scrartcl}
\usepackage{amsmath,amsthm,amssymb}
\usepackage[T1]{fontenc}
\usepackage[utf8]{inputenc}
\usepackage{lmodern}

\usepackage{hyperref}

\title{Berechenbare Analysis \\ SoSe 19}
\author{Benedikt Lüken-Winkels}
\begin{document}

\maketitle
\tableofcontents
\newpage


\section{1. Vorlesung}
\section{2. Vorlesung}

\subsection{Berechenbarkeit}
!NACHHÖREN!


\subsection{Entscheidbarkeit}
\paragraph{Diagonalisierung} Wären die Reellen Zahlen abzählbar, wäre die Diagonalzahl darin enthalten (!Widerspruch).
\begin{table}
  \caption{Diagonialisierungsbeispiel}
  \label{tab:diagonal}
  \begin{center}
    \begin{tabular}{cc}
       a & b \\
       c & d \\
    \end{tabular}
  \end{center}
\end{table}

\paragraph{Definition}
 Menge A Entscheidbar, wenn eine Funktion $ f_{A}(x) $ ,die entscheidet, ob $ x \in A $ berechenbar ist.

\subsection{Berechenbare Reelle Zahlen}

\paragraph{Konstruktive Mathematik}
Formulierung algorithmischen Rechnens: zB $ \exists $ neu definiert als "es existiert ein Algorithmus". Nicht mehr für "klassische Mathematiker" lesbar

\paragraph{Definition}
Für $ x \in \mathbb{R} $ sind die Bedingungen äquivalent (wenn eine Bedingung erfüllt ist, sind alle Erfüllt):
\begin{enumerate}
  \item Eine TM erzeugt eine unendlich lange binäre Representation von $ x $ auf dem Ausgabeband
  \item \textbf{Fehlerabschätzung} Es gibt eine TM, die Approximationen liefert. Formal: $ q:\mathbb{N}\rightarrow \mathbb{Q} $ $ (q_{i})_{i \in \mathbb{N}} $ ist Folge rationaler Zahlen, die gegen $ x $ konvergiert. Bedeutet, dass alle $ q_i $ innerhalb eines bestimmten beliebig kleinen Bereichs um $ x $ liegen. 
  \item \textbf{Intervalschachtelung} Es gibt eine berechenbare Intervallschachtelung: Angabe zweier Folgen rationaler Zahlen mit der Aussage, dass $ x $ dazwischen liegt. Ziel: Abstände von linker und rechter Schranke soll gegen null gehen.
  \item \textbf{Dedekindscher Schnitt}Menge $ \{q \in \mathbb{Q} | q < x \} $ ist entscheidbar. Beispiel $ \sqrt{2} $ ist berechenbar. $ \{ q | q < \sqrt{2} \} = \{ q | q^2 < 2\}$ 
  \item $ z \in \mathbb{Z} $ $ A \subseteq \mathbb{N} $, $ x_A = \sum{i \in A} 2^-1-i $, $ x = z + x_A $
  \item Es exisitert eine Kettenbruchentwicklung
 \end{enumerate}
 \paragraph{Folgerungen}
\begin{itemize}
  \item $ \Rightarrow $ Für Berechenbarkeit muss nur eine der Bedingungen bewiesen werden. Menge der berechenbaren reelen Zahlen = $ \mathbb{R}_c $
  \item Nicht berechenbare reele Zahlen durch Diagonalisierung konstruierbar
  \item $ e $ berechenbar, weil die Fehlerabschätzung (2) existiert
  \item $ \pi $ (Notiert als alternierede Reihe) berechenbar, weil Intervalschachtelung existiert
  \item $ \sqrt{2} $ berechenbar, weil Dedekindscher Schnitt existiert.
 \end{itemize}

\paragraph{Implementierung}
Ziel: zB Berechnung von Differentialgleichungen 


\end{document}
