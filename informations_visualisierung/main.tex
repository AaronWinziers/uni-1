\documentclass[ngerman]{scrartcl}
\usepackage{amsmath,amsthm,amssymb}
\usepackage[T1]{fontenc}
\usepackage[utf8]{inputenc}
\usepackage{lmodern}

\usepackage{hyperref}

\title{Informations Visualisierung \\ SoSe 19}
\author{Benedikt Lüken-Winkels}
\begin{document}

\maketitle
\tableofcontents
\newpage

\section{1. Lecture}
\subsection{Orga}
\begin{itemize}
  \item Website: st.uni-trier.de/lectures/S19/IV/
  \item Tutorial: TBD (beginning: 22.-26.04.)
  \item Final exam: Do, 11.07. (elfths of July) 12-14 (H12)
\end{itemize}

\subsection{Visualisation-Basics}
\begin{itemize}
  \item Combine different kinds of information in one graphic (geographical, temporal, historical, numeric, etc.)
  \item Sharing and visualising abstract data, without physical representation 
  \item Visualisation is not:
  \begin{itemize}
    \item scientific visualisation (non-abstract data)
    \item computer graphics
    \item graphic design
  \end{itemize}
  \item \textbf{Example} Treemap
  \begin{itemize}
    \item representation of a hierarchy of a filesystem
    \item no border used for a square (compression)
    \item light effect shows curvature, indicating where the squares/areas end 
    \item $ \Rightarrow $ only 4 pixels needed instead of 9
    \item Several drawbacks (alternative: tree view)
  \end{itemize}
\end{itemize}
\paragraph{Abstract Data}
\begin{itemize}
  \item Text, table
  \item Hierarchy
  \item Composed data (Multivariate data): Example Napoleon (Slide 1)
  \item Time series: multivariate data with time as a dimension
\end{itemize}
\paragraph*{Definition: Visualisation} 
comprehend and extract data, visualisation produced automatically (not manually by humans)
\paragraph*{Visualisation process} 
\begin{itemize}
  \item graphical user interface
  \item interaction to create and manipulate the visualisation (\textbf{Visual steering})
\end{itemize}

\section{2. Lecture}

\subsection{Diagrams}
\paragraph{Pie charts}
\begin{itemize}
  \item applicable to part-whole relation
  \item Several issues 
  \begin{itemize}
    \item hard to compare values
    \item hard to compare different pie charts
  \end{itemize}
\end{itemize}
\paragraph{Other Diagrams}
\begin{itemize}
  \item Timelines
  \item Sparklines: Reduction to show trend and the change of values over time
\end{itemize}

\subsection{Metaphors and Symbols}
Make constructs/concepts more accessible/imaginable

\subsection{Symbols}
highly simplified representation of objects and acitvities

\paragraph{Isotype} Present quantity/value by number of pictograms

\subsection{Infographics}
\begin{itemize}
  \item Eyecatcher to get people interested in the presented data
  \item Contain few text
  \item Self-explanatory
  \item Should tell a \textbf{story} $ \Rightarrow $ express an opinion
\end{itemize}

\section{3. Lecture}
\subsection{Visual Memory}
\begin{itemize}
  \item The brain fills empty gaps
  \item Distraction by environment (contrast/structure)
  \item $ \Rightarrow $ visual perception is selective
\end{itemize}
\subsection{Visula Information Processing}
3 Phases of processing
\begin{enumerate}
  \item Simple patterns and colors are recognized
  \item Action system: reflexes
  \item Visual working meomry/visual query
\end{enumerate}
\subsection*{Human Eye}
Usage of the properties of visual perception (Anticipation, pattern recognition)
\begin{itemize}
  \item Eye Tracking (works by measuring the reflection form the eye's curvature)
\end{itemize}

\subsection{Color Perception}
3-Color-Theory
\begin{itemize}
  \item Each color consists of rgb
\end{itemize}
Opponent-Color-Theory
\begin{itemize}
  \item After image effect: color-receptors are getting exhausted, so white cannot be 'produced'
  \item three chemical processes with two opponent colors each 
  \item Color is perceived by the difference between the opponent colors
\end{itemize}
$ \Rightarrow $ Color and brightness are relative

\paragraph*{Design Recommendations}
\begin{itemize}
  \item Emphasize with color
  \item Differences with brightness
  \item Coding of categories: max 6 to 12 different colors
  \item Color scales should vary in color and brighntess 
  \item Color perception depends on culture
\end{itemize}


\subsection{Preattentive vision}
\begin{itemize}
  \item Detect patterns before an eye movement
  \item Motion is preattentive
  \item $ \Rightarrow $ Use preattentive patterns to encode information (spot an outlier)
\end{itemize}

\subsection{Pattern Recognition}
\begin{itemize}
  \item Edge detection
  \item Simple patterns (detect small distortions)
  \item Complex patterns
  \item Object recognition (compare observation with learned patterns to recognise an object)
\end{itemize}


\subsection{Motion recognition}
Different elements perform similar motions
\begin{itemize}
  \item Recognize patterns to identify object
  \item Recognize change after each frame
  \item Movements seem related, when they are in synch
  \item $ \Rightarrow $ Indicate a relation with a synchronous animation 
  \item Motion can induce causality
\end{itemize}


\end{document}






